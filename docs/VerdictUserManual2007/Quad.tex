\chapter{Quadrilateral Quality Metrics}

All the metrics in this section are defined on a quadrilateral element with vertices
shown in Figure~\ref{f:quad}. Furthermore, we define the following edge vectors for
convenience. Note that each edge has two versions, one defined by its endpoints and
another indexed by sequential integers:
\begin{equation*}
\begin{array}{lcl}
\vec L_0 &=& \vec P_1 - \vec P_0\\
\vec L_1 &=& \vec P_2 - \vec P_1\\
\vec L_2 &=& \vec P_3 - \vec P_2\\
\vec L_3 &=& \vec P_0 - \vec P_3
\end{array}\rule{10em}{0pt}
\begin{array}{lcl}
\vec L_{01} &=& \vec P_1 - \vec P_0\\
\vec L_{12} &=& \vec P_2 - \vec P_1\\
\vec L_{23} &=& \vec P_3 - \vec P_2\\
\vec L_{30} &=& \vec P_0 - \vec P_3.
\end{array}
\end{equation*}

The quadrangle edge lengths are denoted as follows:
\[
L_0 = \normvec{L_0}\quad
L_1 = \normvec{L_1}\quad
L_2 = \normvec{L_2}\quad
L_3 = \normvec{L_3}
\]
and the largest and smallest edge lenghts are, respectively,
\[
L_{\min} = \min\left(L_0, L_1, L_2, L_3\right)
  \rule{2em}{0pt}
L_{\max} = \max\left(L_0, L_1, L_2, L_3\right)
\]

The diagonals of a quadrilateral are denoted
\begin{equation*}
\begin{array}{lcl}
\vec D_0 &=& \vec P_2 - \vec P_0
\end{array}\rule{10em}{0pt}
\begin{array}{lcl}
\vec D_1 &=& \vec P_3 - \vec P_1
\end{array}
\end{equation*}
and the longest diagonal has length
\[
D_{\max} = \max\left\{ \normvec{D_0}, \normvec{D_1} \right\}.
\]

\begin{figure}
  \centering
  \begin{subfigure}[b]{0.49\textwidth}
    \centering
    \includegraphics[width=2in]{quad}
    \caption{Vertices of a quadrilateral.}
  \end{subfigure}
  \hfill
  \begin{subfigure}[b]{0.49\textwidth}
    \centering
    \includegraphics[width=2in]{quad-axes}
    \caption{Principal axis vectors.}
  \end{subfigure}
  \caption{A quadrilateral showing notation used in metric definitions.\label{f:quad}}
\end{figure}

The principal axes are
\begin{equation*}
\begin{array}{lcl}
\vec X_1 &=& \left(\vec P_1 - \vec P_0\right) + \left(\vec P_2 - \vec P_3\right)\\
\vec X_2 &=& \left(\vec P_2 - \vec P_1\right) + \left(\vec P_3 - \vec P_0\right)
\end{array}
\end{equation*}
and the cross derivatives of the map from parametric to world space are oriented along
\begin{equation*}
\begin{array}{lcl}
\vec X_{12} &=& \left(\vec P_0 - \vec P_1\right) + \left(\vec P_2 - \vec P_3\right) =\\
\vec X_{21} &=& \left(\vec P_0 - \vec P_3\right) + \left(\vec P_2 - \vec P_1\right).
\end{array}
\end{equation*}

Each corner has a normal vector associated with it
\begin{equation*}
\begin{array}{lcl}
\vec N_0 &=& \vec L_3 \times \vec L_0\\
\vec N_1 &=& \vec L_0 \times \vec L_1
\end{array}
\rule{10em}{0pt}
\begin{array}{lcl}
\vec N_2 &=& \vec L_1 \times \vec L_2\\
\vec N_3 &=& \vec L_2 \times \vec L_3
\end{array}
\end{equation*}
and these vectors can be normalized to unit length:
\begin{equation*}
\begin{array}{lcl}
\hat n_0 &=& \dfrac{\vec N_0}{\normvec{ N_0}}\\
\hat n_1 &=& \dfrac{\vec N_1}{\normvec{ N_1}}
\end{array}
\rule{10em}{0pt}
\begin{array}{lcl}
\hat n_2 = \dfrac{\vec N_2}{\normvec{ N_2}}\\
\hat n_3 = \dfrac{\vec N_3}{\normvec{ N_3}}.
\end{array}
\end{equation*}

In addition to corner normals, we can define a ``center'' normal
\begin{equation*}
\vec N_{c} = \vec X_1 \times \vec X_2
\end{equation*}
and its unit-length companion
\begin{equation*}
\hat n_{c} = \frac{\vec N_{c}}{\normvec{ N_{c}}}
\end{equation*}
In the event that the vertices of the quadrilateral are all
contained in the same plane, all the unit normals will be
equivalent (i.e., $\hat n_0 = \hat n_1 = \hat n_2 = \hat n_3 = \hat n_c$).

\begin{figure}[htb]
  \centering
  \includegraphics[width=2in]{quad-vertex-areas}
  \caption{Areas associated with each quadrilateral vertex.%
                                                    \label{f:quad-vertex-areas}}
\end{figure}

It is often useful to partition the quadrilateral into four areas, one
associated with each vertex. These areas are denoted
\begin{equation*}
\alpha_k = \hat n_c \cdot \vec N_k\rule{10em}{0pt}\forall k\in\{0,1,2,3\}
\end{equation*}
and are shown in Figure~\ref{f:quad-vertex-areas}.
If $\vec N_c = \vec 0$, then the signed corner areas are undefined,
and all the metrics which depend on $\alpha_k$ are undefined.
In this case, we set $\alpha_k = 0$ for $k=0,1,2,3$.
When $\alpha_k \leq 0$ for any one or more $k$, the quadrilateral
is degenerate.
This occurs when
an element is so small its edge length approach the machine epsilon or
when its vertices are collinear or
when its vertices define a concave quadrilateral.

% -------------------Metric Table-------------------
\newcommand{\quadmetrictable}[8]{%
  \begin{center}
  \begin{tabular}{ll}
    \multicolumn{2}{r}{\textbf{\sffamily\Large quadrilateral #1}}\\\hline
    Dimension:             & #2\\ 
    Acceptable Range:      & #3\\ 
    Normal Range:          & #4\\ 
    Full Range:            & #5\\ 
    $q$ for unit square:   & #6\\
    Reference:             & #7\\
    \verd\ function:       & \texttt{#8}\\ \hline
  \end{tabular} 
  \end{center}
}

\newpage \input{QuadArea}
\newpage \input{QuadAspectRatio}
\newpage \input{QuadCondition}
\newpage \input{QuadDistortion}
\newpage \input{QuadEdgeRatio}
\newpage \input{QuadJacobian}
\newpage \input{QuadMaxAspectFrobenius}
\newpage \input{QuadMaximumAngle}
\newpage \input{QuadMaximumEdgeRatio}
\newpage \input{QuadMedAspectFrobenius}
\newpage \input{QuadMinimumAngle}
\newpage \input{QuadOddy}
\newpage \input{QuadRadiusRatio}
\newpage \input{QuadRelativeSizeSquared}
\newpage \input{QuadScaledJacobian}
\newpage \input{QuadShape}
\newpage \input{QuadShapeAndSize}
\newpage \input{QuadShear}
\newpage \input{QuadShearAndSize}
\newpage %---------------------------Skew-----------------------------
\section{Skew}

First define normalized principal axes
\[
\begin{array}{lcl}
\hat X_1 &=& \frac {\vec X_1} {\normvec{X_1}}\\
\hat X_2 &=& \frac {\vec X_2} {\normvec{X_2}}.
\end{array}
\]

The skew is then
\[
q = | \hat X_1 \cdot \hat X_2 |.
\]
A geometric intepretation of the skew is that it measures the angle between the principal axes.
In fact, it is the absolute value of the cosine of the angle between the principal axes.

Note that if $\normvec{X_1}$ or $\normvec{X_2} < DBL\_MIN$, we set $q = 0$.

\quadmetrictable{skew}%
{$1$}%                                      Dimension
{$[0,0.5]$}%                                Acceptable range
{$[0,1]$}%                                  Normal range
{$[0,1]$}%                                  Full range
{$0$}%                                      Unit square
{Adapted from \cite{rob:87}}%               Citation
{v\_quad\_skew}%                            Verdict function name


\newpage \input{QuadStretch}
\newpage \input{QuadTaper}
\newpage %---------------------------Warpage-----------------------------
\section{Warpage}

Warpage is defined as
\[
q =
  1 - \min \left\{
    \left( \hat n_0 \cdot \hat n_2  \right)^3,
    \left( \hat n_1 \cdot \hat n_3  \right)^3
  \right\}
\]
which is the cosine of the minimum dihedral angle formed by
planes intersecting in diagonals (to the fourth power).

Note that if $\normvec{N_k} < DBL\_MIN$ for any $k$, we set $q = DBL\_MAX$.

\quadmetrictable{warpage}%
{$1$}%                                      Dimension
{$[0.3,1]$}%                                Acceptable range
{$[-1,1]$}%                                 Normal range
{$[-DBL_MAX,DBL\_MAX]$}%                    Full range
{$0$}%                                      Unit square
{--}%                                       Citation
{v\_quad\_warpage}%                         Verdict function name




